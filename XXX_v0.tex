\documentclass[preprint]{elsarticle}

\usepackage{amsmath,amssymb}
\usepackage{tikz}
\usepackage{natbib}
\usepackage{amssymb}
\usepackage{multirow}
\usepackage{subfigure}
\usepackage{xcolor}
\usepackage{setspace}
\usepackage{lipsum}  
\usepackage{lineno,hyperref}
\modulolinenumbers[5]
\journal{The journal of your choosing}


\begin{document}

%%%%%%%%%%%%%%%%% FRONT MATTER %%%%%%%%%%%%%%%%%%%%%%%%%%%%%%%%%
% Edit as you see fit

\begin{frontmatter}

\title{manuscripTEX}

\author[U]{Random name1}
\ead{blah@blah.X}
\author[U]{Random name2}
\author[I]{Random name3}
\author[U]{Random name4\corref{mycorrespondingauthor}}
\cortext[mycorrespondingauthor]{Corresponding author}
\ead{XXX@XXX.edu}

\address[U]{University, Utopia}
\address[I]{National Lab, Utopia}

\begin{abstract}
TBD.
\end{abstract}

\begin{keyword}
keyword1 \sep keyword2 \sep keyword3
\end{keyword}

\end{frontmatter}

% To begin line #:
\linenumbers

% Set your graphics path:
\graphicspath{ {figures/} }
%%%%%%%%%%%%%%%%%%%%%%%%%%%%%%%%%%%%%%%%%%%%%%%%%%%%%%


\section{Before you begin to edit}

\begin{enumerate}[(1)]
\item Rename XXX to something meaningful - say YYY.
\item Update XXX in makefile - the meaningful YYY.
\item Update BIBMASTER in makefile. This is the absolute path to your master .bib file.
\item Edit YYY.tex as you see fit. For now, I recommend editing just the basics- working title, author list, etc.
\item To create YYY.pdf, open the terminal and type \emph{make pdf}. This will create YYY.pdf but also create a bunch of other quasi-importan files. Your pdf does not have citations yet. To do this you must \emph{make bib}. Now you have generated a list of citations but they have not yet been added to the pdf. \emph{make pdf} to add the citations. You still dont have cross referencing yet. \emph{make pdf} again. Now you have all sorts of extra files you dont really know what to do with. \emph{make clean} to remove these files. If all this seems tiring just type \emph{make} to excecute the entire workflow (latex+bib+crossref+clean) all at once.
\item Congratulations you have just compiled a tex file! Now would be a good time to checkpoint (small) progress you have made. Open the terminal and type \emph{make snapshot}. This will create v1.
\item When you get back to it- edit v1 to your heart's extent.
\item After you`ve made enough changes and want to create a checkpoint for yourself (or you want to call it a day and come back later). 
\item Only then you can begin editing v1.

\item To create a markup, i.e. highlight differences between current and previous version at any point- make markup. 
\item If you wish to checkpoint your progress and/or share, \emph{make markup} at any point in time. This will make a backup and automatically create a new version of the file to begin editing.
\item When you create a snapshot. 
\end{enumerate}

\section{Experiments}

Any guesses what equation \ref{eq:HK} is?
\begin{equation} \label{eq:HK}
\dot{m}^{''}=\sqrt{\frac{k_b}{2 \pi m}}\left(\alpha_e\rho^V_{sat}(T^L_i)\sqrt{T_i^L}-\alpha_c\rho^V\sqrt{T^V}\right)
\end{equation}

A test image (figure \ref{f1}).
\begin{figure} [htb]
\centering
\includegraphics[width=2in]{university_logo}
\caption{Go Huskies! \label{f1}}
\end{figure}


\section{Citation}
Random citation / selfless self-advertisement \cite{bellur_2018a}.

\section{Conclusion}
No template is ever complete without \emph{lorem ipsum}:

\noindent \lipsum[1]

%%%%%%%%%%%%%%%%%%%%%%%
%% `Elsevier LaTeX' style
\bibliographystyle{elsarticle-num}
\biboptions{comma,sort&compress}
\bibliography{XXX}

\end{document}

